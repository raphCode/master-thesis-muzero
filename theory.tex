\chapter{Theory}
% Ich finde ein extra Theorie Kapitel gut, in dem man die theoretischen Grundlagend
% zusammenfasst, die nötig sind um die verwendeten Algorithmen zu verstehen. 
% In Research-papern ist hierfür meist kein Platz aber in einer Abschlussarbeit kann man
% damit gut zeigen, dass man das Material verstanden hat.

\begin{itemize}
    \item Original Paper has special requirements for the type of games
    that can be learned:
    \todo{Check if the \cite{muzero} mentions these restrictions, I actually found
    these in \url{https://sebastianbodenstein.net/post/alphazero/}}
    \question{please comment on the general situation: "found information / idea on
    the internet, should I mention the link or just confirm the information exists in
    the original paper and only reference that?}
    \begin{itemize}
        \item sequential (informaticup game is not)
        \item single player or two player
        \item two player: zero sum (optimisations in the search tree:
        flipping rewards every other turn)
        \item transitive "better than" relation on game strategies: unique best
        strategy must exist (counterexample: starcraft)
    \end{itemize}
    \item Basic techniques:
    \begin{itemize}
        \item Monte Carlo Tree Search \cite{mcts_survey}
        \item Model-based RL
    \end{itemize}
\end{itemize}
