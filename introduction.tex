\chapter{Introduction}
% Hier beschreibst du die Idee und Motivation deiner Arbeit. 
% Ziel ist es zu sagen was du gemacht hast und warum es wichtig ist, dass es jemand
% gemacht hat.
% Es sollte reichen die Introduction und Conclusion zu lesen, um vollständig zu verstehen
% was du gemacht hast und was rausgekommen ist (ohne details natürlich).
% Laut Prof. André eines der wichtigesten Kapitel.

Games always have been an intesting playground for artifical intelligence because of the
availabilty of many different types with a wide range of difficulties.
Also, machine performance can esily be compared to expert players, providing a human
baseline.
Furthermore, machine learning algorithms also often use games as their environments
because their requirement for huge amounts of training data can easily be satisfied by
simulating any number of playouts on a computer.

Different games require problem solving in distinct areas, so it was common to have
specialized algorithms for each game.
\question[noinline]{cite here?}
With the advent of reinforcement learning, techniques were discovered that perform well
when applied to a broader range of games.
\todo[noinline]{cite}
One algorithm that was especially successful in generalizing across multiple complex games
was MuZero\cite{muzero} by \citeauthor{muzero}.

This paper revisits the MuZero algorithm and evaluates its performance on different games,
ranging from a classic atari game to more challenging environments that do not exactly fit
the problem domain originally intended by the authors.
This includes a game that requires cooperation with other players in order to win and a
round-based game which can be considered incomplete information because the moves of other
agents are only revealed after all made their choice.
\todo{vey often the word game?}

